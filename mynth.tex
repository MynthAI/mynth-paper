\hypertarget{disclaimer}{%
\section{Disclaimer}\label{disclaimer}}

This white paper is for informational purposes only and should not be
considered as financial advice or an offer, solicitation, or
recommendation to buy or sell any securities, tokens, or other financial
instruments. The information provided here is on an ``as is'' basis
without any warranties or representations, either expressed or implied.

The authors, contributors, and the team do not accept any liability or
responsibility for any loss or damage of any kind resulting directly or
indirectly from using or relying on the information in this white paper.
Readers are advised to conduct their own research and consult with
professional advisors before making any investment decisions.

By accessing this white paper, you acknowledge and agree that you have
read, understood, and accepted the terms of this legal disclaimer.

\hypertarget{abstract}{%
\section{Abstract}\label{abstract}}

Mynth is an innovative synthetics swap platform that enables seamless
conversion of synthetic assets to real-world assets. Synthetic assets
can offer lucrative arbitrage opportunities but utilizing them is
challenging due to their complexity. Mynth revolutionizes this process
by providing a user-friendly mechanism to swap synthetic tokens for
their real counterparts. With Mynth's automatic arbitrage feature, users
can effortlessly capitalize on these opportunities without the need for
advanced trading knowledge or strategies.

\hypertarget{synthetic-assets}{%
\section{Synthetic Assets}\label{synthetic-assets}}

Synthetic assets are financial instruments that allow users to gain
exposure to various real-world assets without owning the underlying
asset. They mimic the performance of traditional assets using financial
tools like derivatives or debt, making them more accessible and
cost-effective for investors.

Blockchain technology revolutionizes the synthetic asset market by
offering transparency, efficiency, low barriers to entry, and
decentralization. Synthetic assets can be easily created, traded, and
settled on decentralized platforms, fostering a more inclusive financial
ecosystem and equalizing the playing field.

An example of a synthetic asset on the Cardano blockchain is iUSD from
the \href{https://indigoprotocol.io/}{Indigo Protocol}. iUSD is a
stablecoin that mirrors the value of the US dollar, giving users
exposure to the world's most widely used currency without holding
physical (or digital) dollars. Blockchain technology enables seamless
trading and transferring of iUSD across the globe, unlocking new
investment opportunities for users worldwide.

\hypertarget{problems-with-synthetic-assets}{%
\section{Problems with Synthetic
Assets}\label{problems-with-synthetic-assets}}

Synthetic assets, although innovative and useful in many cases, do have
their limitations. One main issue is their inapplicability in real-world
situations. Unlike traditional assets, synthetic assets on the
blockchain only exist within their original ecosystem, which restricts
their practical uses.

Another challenge associated with synthetic assets is the potential for
temporary depegging scenarios. In these instances, the synthetic asset's
price may greatly differ from the real-world counterpart's price. This
inconsistency can lead to confusion and uncertainty for traders and
investors, as the synthetic asset's value might not accurately reflect
the actual asset's worth.

The complexity of synthetic assets can also create difficulties for
users. Many people may not fully understand how to use synthetic assets
or take advantage of the arbitrage opportunities offered by synthetic
platforms. This lack of knowledge can result in missed opportunities and
potential losses for those who are not familiar with the complexities of
trading synthetic assets.

\hypertarget{solution}{%
\section{Solution}\label{solution}}

Mynth offers a smooth, user-friendly platform for exchanging synthetic
assets with their real-world equivalents. By connecting synthetic and
real assets, Mynth enables users to enjoy the advantages of synthetic
assets while easily converting them into tangible assets when necessary.
This improves the overall usefulness and relevance of synthetic assets
in everyday transactions and investment situations.

Temporary depegging events can cause significant changes in synthetic
asset values, leading to instability and unpredictability for users.
Mynth provides an effective and dependable method for swapping assets,
contributing to the stabilization of synthetic assets. The platform's
non-custodial and fast exchange process ensures that users can securely
and swiftly convert their synthetic assets into real-world assets,
reducing the effects of temporary depegging events and preserving the
stability and value of their investments.

Mynth streamlines the management and conversion of synthetic assets by
handling the complex arbitrage and conversion process for users. By
managing the intricate aspects of asset conversion and offering a simple
method for asset swapping, Mynth removes the challenges and difficulties
often linked to synthetic asset management, making it more accessible
and attractive to a broader range of users.

\hypertarget{how-it-works}{%
\section{How it Works}\label{how-it-works}}

Mynth is a platform that enables users to trade synthetic assets for
their respective non-synthetic counterparts. The process is
straightforward and user-friendly, requiring only a few steps to
complete a transaction.

\begin{enumerate}
\item
  Users select the synthetic asset they want to exchange.
\item
  Users provide the receiving address for the non-synthetic asset they
  wish to obtain. Mynth will use this address to send the exchanged
  asset after the transaction is complete.
\end{enumerate}

Mynth maintains a pool of assets to facilitate these transactions. When
a user initiates a trade, Mynth draws from this pool to send the desired
non-synthetic asset to the user's provided address. This process ensures
quick and efficient transactions.

Behind the scenes, Mynth employs a team of professional arbitrage
traders who manage asset conversions with the help of artificial
intelligence. These traders and automated intelligent systems replenish
the asset pools by taking advantage of price differences in various
markets. This ensures that Mynth can continue facilitating trades for
its users.

\hypertarget{trustless-cross-chain-swaps}{%
\subsection{Trustless Cross-chain
Swaps}\label{trustless-cross-chain-swaps}}

A common issue with centralized exchanges is the need for users to trust
that the exchange will execute the swap as promised. Cryptocurrency
technology aims to eliminate trust and instead focus on verification.
Mynth enables users to swap assets between its liquidity pools across
blockchains without trusting centralized parties. The implementation
consists of two versions: the version uses Cardano multisig wallets, and
the second version employs Cardano decentralized oracles. Initially
Mynth will launch using the first version, and then later is intended to
be updated to use the second version.

In version one, users deposit assets into a time-locked 3 out of 5
multisig wallet. This means that either 3 out of 5 signatures are
required to unlock the tokens, or the user can retrieve the tokens after
the time- lock expires. Users can choose the verifiers for each swap,
with Mynth's user interface selecting verifiers by default. Verifiers are
responsible for confirming transactions on non-Cardano blockchains. To
unlock a token, the verifier checks if a transaction occurred. If the
verifier agrees, then unlocking is allowed, and the verifier returns a
signature. If 3 out of 5 verifiers return valid signatures, the token can
be unlocked and transferred to Mynth's Cardano liquidity pool.

For example, a user wants to swap iUSD on Cardano for USDT on Tron using
Mynth. The user initiates the swap through Mynth's application, and
metadata is embedded into the Cardano transaction containing necessary
information for a third party to validate the swap. The iUSD is sent
from the user's wallet to a Cardano multisig wallet. After validation on
the Cardano blockchain, Mynth transfers USDT to the user's Tron wallet.
Mynth then calls upon verifiers to unlock the iUSD on Cardano and
transfer it to Mynth's Cardano iUSD liquidity pool. Verifiers check the
Tron transaction and return signatures to permit the unlock. In the end,
the user has iUSD withdrawn from their Cardano wallet and USDT deposited
into their Tron wallet, while Mynth has iUSD deposited into their
Cardano liquidity pool and USDT withdrawn from their Tron liquidity
pool.

In version two, Mynth uses Cardano oracles like Charli3 or Orcfax to
verify transactions on non-Cardano blockchains. Oracle nodes
periodically update data on the Cardano blockchain, and if consensus is
reached among the nodes, the data is considered valid. Users initiate
swaps on Cardano and deposit tokens into a Cardano smart contract.
Tokens remain locked in the smart contract until on-chain data verifies
the tokens' transfer on the non-Cardano blockchain.

Oracle nodes cross-reference locked tokens on Cardano with transactions
on non-Cardano blockchains. If the cross-reference is valid, the oracle
node marks the Cardano input transaction as valid. All validated
transactions are concatenated, hashed, and stored on-chain. To unlock
the tokens, the Cardano smart contract checks the oracle on-chain data
and compares the hashes. If the hashes match, the unlock is validated
and permitted, allowing the transfer of tokens to Mynth's Cardano
liquidity pool.

For example, a user wants to swap iUSD on Cardano for USDT on Tron using
Mynth. The user initiates the swap through Mynth's application,
transferring their iUSD into Mynth's smart contract. Oracles
periodically update on-chain data. When the USDT is transferred to the
user's wallet, the oracle marks the transaction as valid during its
update process. Mynth then executes the smart contract's unlock feature
to transfer the iUSD to Mynth's Cardano liquidity pool. In the end, the
user has iUSD withdrawn from their Cardano wallet and USDT deposited
into their Tron wallet, while Mynth has iUSD deposited into their
Cardano liquidity pool and USDT withdrawn from their Tron liquidity
pool.

\hypertarget{swap-limits}{%
\section{Swap Limits}\label{swap-limits}}

Mynth will enforce minimum and maximum limits on swap sizes. The minimum
swap size will be \$100 equivalent in value, whereas the maximum swap
size will be \$10,000 equivalent in value. If Mynth's asset supplies
become low, temporary restrictions might be applied to swaps until the
supply is restored.

\hypertarget{mynth-token}{%
\section{Mynth Token}\label{mynth-token}}

Mynth Token (MNT) is a Cardano native token with a fixed total supply of
100,000,000. It serves as a utility token within the Mynth ecosystem,
allowing users to pay for swap fees when exchanging assets on the
platform.

Users can stake MNT. MNT stakers will gain access to the Mynth DAO,
enabling them to vote on decisions related to the platform's development
and governance, such as enabling profit sharing of Mynth's collected
fees.

\hypertarget{mynth-fee}{%
\subsection{Mynth Fee}\label{mynth-fee}}

The Mynth Fee is a transaction cost applied to users when they perform a
swap on the Mynth platform. The fee calculation is based on two factors:
a percentage of the transaction value and a minimum fee in MNT tokens.
The fee will be equivalent to 1\% of the value of the exchange or 1 MNT,
whichever value is higher.

For example, let's consider a user who wants to swap 1,000 iUSD on the
platform and a scenario where MNT is trading at \$0.01. In this case,
the fee will be calculated as 1\% of 1,000 iUSD, which equals \$10.
Since the value of \$10 is higher than the minimum fee of 1 MNT
(\$0.01), the user will be charged \$10 worth of MNT for this
transaction (i.e., 1,000 MNT).

To illustrate the minimum fee scenario, let's assume a user is
exchanging an amount equivalent to \$50, and MNT is trading at \$1. In
this case, the fee would be 1\% of \$50, which is \$0.50. However, since
the minimum fee of 1 MNT (\$1) is higher than \$0.50, the user will be
charged 1 MNT for the transaction instead.

The MNT token fee collected during swaps is burned and permanently
removed from the circulating supply. This process reduces the overall
supply over time.

\hypertarget{mynth-tokenomics}{%
\subsection{Mynth Tokenomics}\label{mynth-tokenomics}}

MNT has a total supply of 100,000,000 tokens. The distribution is as
follows:

\begin{enumerate}
\item
  \textbf{Mynth DAO Treasury (50\%):} 50,000,000 MNT is locked in the
  Mynth DAO Treasury to ensure the long-term sustainability of the
  project.
\item
  \textbf{Core Mynth Team (25\%):} 25,000,000 MNT is allocated to the
  core Mynth team. This allocation vests over a 3-year period to
  encourage the team's dedication to the project. Vesting takes place
  monthly on the last day of each month, having started the month after
  MNT was minted.
\item
  \textbf{Mynth Funding Initiatives (10\%):} 10,000,000 MNT is allocated
  to Mynth funding initiatives and expenses, such as partnerships,
  marketing, development, trading, and fundraising, to support the
  growth and development of the ecosystem.
\item
  \textbf{Mynth Airdrop (10\%):} 10,000,000 MNT is allocated to the
  Mynth airdrop, allowing the community to participate and receive free
  swaps.
\item
  \textbf{Mynth Protocol Owned Liquidity (5\%):} 5,000,000 MNT is
  allocated to MNT liquidity on DEXs and locked in the Mynth DAO
  Treasury, allowing MNT to be available to the public. This
  distribution plan ensures a balanced allocation of MNT tokens,
  supporting the long-term growth and success of the Mynth project.
\end{enumerate}

This distribution plan ensures a balanced allocation of MNT tokens,
supporting the long-term growth and success of the Mynth project.

\hypertarget{mynth-v1}{%
\section{Mynth v1}\label{mynth-v1}}

Version one of Mynth is designed to be a simple and focused platform
with a limited feature set. The primary function of this initial version
is to facilitate the swapping of iUSD to USDT on the Tron blockchain. By
concentrating on this specific use case, the Mynth team aims to
determine product-market fit, identify potential users, establish a
strong community, and lay the groundwork for the Mynth ecosystem.

This first iteration of Mynth serves as a foundation for future
development and expansion. As the platform evolves, additional synthetic
assets will be supported, expanding the range of trading options for
users. Moreover, Mynth will extend its reach to other blockchains,
broadening its appeal and utility across the cryptocurrency landscape.

\hypertarget{mynth-roadmap-to-bridge-the-gap-between-synthetics-and-the-real-world}{%
\section{Mynth Roadmap to Bridge the Gap Between Synthetics and the
Real-World}\label{mynth-roadmap-to-bridge-the-gap-between-synthetics-and-the-real-world}}

Mynth aims to bridge the gap between synthetics and the real world,
focusing primarily on integrating the Indigo Protocol---the largest
synthetics protocol on Cardano---and connecting the Cardano and Tron
blockchains. Below is a roadmap outlining the key milestones for Mynth's
development and growth.

\hypertarget{phase-1-initial-focus}{%
\subsection{Phase 1: Initial Focus}\label{phase-1-initial-focus}}

During the initial phase, Mynth will focus on connecting the Cardano and
Tron blockchains. This will be achieved by tailoring the Indigo
Protocol, enabling seamless interaction between the two ecosystems. This
crucial step lays the foundation for Mynth's future endeavors.

\hypertarget{phase-2-post-launch}{%
\subsection{Phase 2: Post-Launch}\label{phase-2-post-launch}}

Following a successful launch, the Mynth team will shift their attention
to supporting swaps between iBTC and BTC, as well as iETH and ETH. This
will allow users to easily exchange synthetic assets for their native
counterparts, increasing liquidity and promoting greater adoption of the
Mynth and Indigo platforms.

\hypertarget{phase-3-expansion-and-growth}{%
\subsection{Phase 3: Expansion and
Growth}\label{phase-3-expansion-and-growth}}

As Mynth gains traction and user adoption, the team will explore
additional opportunities to expand the platform's capabilities.
Potential expansions include integrating more blockchain networks,
supporting a wider range of synthetic assets, and developing new
features to enhance the overall user experience.

\hypertarget{why-usdt-and-tron}{%
\section{Why USDT and Tron?}\label{why-usdt-and-tron}}

USDT and Tron are chosen for their unique advantages and strong
compatibility with the DeFi ecosystem. Tron, as the second-largest DeFi
blockchain by TVL, offers a more accessible development environment
compared to Ethereum. Its low transaction costs and wide-ranging
integrations make it an attractive platform for developers and users
alike.

Tron's established connection with a banking partner used by the Mynth
team simplifies arbitrage execution. Furthermore, Tron's support across
various exchanges in the Americas, Europe, and Asia ensures a global
reach and increased adoption potential.

USDT, the most popular stablecoin, offers many exchange and fiat
integrations. This allows users to effortlessly transfer USD to their
bank accounts. Although alternatives like USDC or TUSD could be
considered, they currently lack the infrastructure integrations that
USDT provides. This gives USDT an advantage in terms of accessibility
and ease of use.

\hypertarget{summary}{%
\section{Summary}\label{summary}}

Mynth is a new platform that addresses the challenges and limitations
associated with synthetic assets. It provides a user-friendly mechanism
to swap synthetic tokens for their real-world counterparts. By
connecting synthetic and real assets, Mynth allows users to capitalize
on the benefits of synthetic assets while ensuring their practical use
in everyday transactions and investments. As Mynth continues to develop,
it will support a growing range of synthetic assets and expand to other
blockchains, bridging the gap between synthetic and real-world asset
markets. With its innovative approach, Mynth aims to transform the
synthetic asset landscape and create a more inclusive, accessible
financial ecosystem for users worldwide.
